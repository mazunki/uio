\documentclass{article}
\usepackage{amssymb}

\title{EXPHIL03E --- Kant}
\begin{document}

\section{Categoritve imperative}
It will allow you test if you are morally allowed to do something

\section{The good will} 
Spending their whole life helping people
From duty if we know the motive is not love 

even the person who loves humanity is not morally ``correct''

love is an inclination, morality must come from the rationale, aka the good will.

\textbf{Claim1}: Only actions from duty has moral worth. Will must be moved by duty.

\textbf{Claim2}: Purpose is irrelevant, but we care about the maxim (the rule)

Motive $\rightarrow$ Action $\rightarrow$ consequences (consequentialism) \\
$\hspace*{1em}\swarrow\hspace*{2em}\searrow$\\
Duty, inclination \\

Maxim is a rule, you can test if the rule is \emph{inclination} or \emph{duty}.

\textbf{Claim3}: Respect for the moral laws (not the country's laws).
Respect is not an inclination.

\section{Moral reasoning}
Moral laws are a science which come from reasoning, and those are universal truths of what is right or wrong.
(No cultural relativism, anyone can get it right (although not everyone might))

Generalize from actions. All actions come from an implicit maxim.
Use maxims to create the law. \\

It doesn't matter what X, Y, or Z are. Z should be good:
\begin{itemize}
\item Action: In situaton X, I consider Y to get Z.
\item Maxim: Whenever in situation X, I do Y to get Z (subjective)
\item Law: Whenever anyone is in X, they should do Y to get Z (objective, requires testing)
\end{itemize}
Golden rule (do to others what they want them to do to you) $\rightarrow$ Only do things that can be universalized.

\pagebreak
\section{Categorical imperative}
What is a law? It commands. Do this, don't do that.

Imperative (\emph{Do X!})\\
$\hspace*{1em}\swarrow\hspace*{3em}\searrow$\\
Hypothetical, Categorical
(If...then), (One must, regardless of desires)\\

\section{Humanity and Autonomy}
Decision procedures for reaching universal laws
\begin{itemize}
\item The formula of universal law
\item The formula of humanity
\item The formula of the kingdom of ends
\end{itemize}

\subsection{Formula of universal law}
Consider an action $\rightarrow$ Formulate a maxim $\rightarrow$ Formulate a universal law $\rightarrow$ Imagine a World in which Everyone Always Follows this Law $\rightarrow$ Is there a contradiction in conception, that is, would the maxim ``necessarily destroy itself'' when made a universal law?

If yes, the opposite of the law is a duty. The law is also forbidden.

If no, the action is permissible. (not meaning it is necessarily morally correct, just allowed)

\subsubsection{Example}
Should I make a false promise in order to get out of some difficulty $\rightarrow$ Whenever I am in difficulty, I may make a false promise in order to get out of it $\rightarrow$ Whenever anyone is in difficulties, they should make a false promise in order to get out of it $\rightarrow$ Imagine a World in which everyone always makes false promises to get out of difficulties $\rightarrow$ Is there a contradiction in conception?

Yes: in such a world, it makes no sense to lie to get out of difficulties, since no one would ever believe anyone else's promises

\subsection{The formula of humanity}
Treat other rational beings as rational beings (as ends), but never as a means (for your own interest).

\subsubsection{Example}
Asking someone the time, so they can act with their rational agency, instead of taking them by the wrist, where one'd solely use them as a means.

\subsection{Autonomy}

Giving laws to oneself.
All rational beings are subjects and legislators simultaneously, so they give laws but also have to follow them. All rational beings will reach the same truths due to its objectivity.

We must respect the rational agency and autonomy of others.

\section{Anthropology and the nature of women}

Women are rational beings, but their relationship is in terms of men and the survival of the species, and cultivation/refinement \emph{(Jane Austen, dancing, singing...)} of society.

Masters of the home. Politically subordinate to men.

\end{document}