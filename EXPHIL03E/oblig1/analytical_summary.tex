
\documentclass{assignment}
\usepackage[utf8]{inputenc}
\usepackage{lmodern}

\begin{document}
\begin{problem}
Provide an analytic summary of Descartes’s example involving the piece of wax (Descartes 1 2017, Meditation 2, section 11 and 12), and give an account of Descartes’s conclusion, that “the wax is solely known through the mind”.%%wcignore

Length: 600 words max.%%wcignore
\end{problem}

\vspace*{2\baselineskip}
\begin{answer}
During René Descartes second meditation, the philosopher argued about the definition of a wax. In context, he was trying to define the meaning of things (bodies). His definition of a body until now is \emph{something which can be in a place, has a volume by itself, can be percieved through the senses, and is movable}.%54

He had started by trying to define who Himself was. Since it is too difficult to define a person, due to its complexity, he moved forward and tried to define a piece of wax.%34

\vspace*{1\baselineskip}
He had chosen a piece of wax since he considered that he distinctively knew it, being a \emph{particular} body which he \emph{could touch and see}. By giving the piece of wax different physical properties such as colour, figure and size; he learned after a while that these properties would not be permanent, but instead change over time ---due to the heat of the fire---.%62

Since the definition of the wax couldn't be given by the properties he had witnessed, Descartes wondered if we were still talking about the same piece of wax. He concluded that the piece we were talking about hadn't changed, but instead stated that one cannot define an item by the means of the senses. Sensorial perceptions would only explain how they appear to any given person at any given time.%49




\end{answer}

\end{document}