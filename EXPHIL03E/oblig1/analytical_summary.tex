
\documentclass{myassignment}

\exercisesheet{Obliger 1}{Analytic Text}
\courselabel{EXPHIL03E}

\begin{document}
\begin{problem}
Provide an analytic summary of Descartes’s example involving the piece of wax (Descartes 1 2017, Meditation 2, section 11 and 12), and give an account of Descartes’s conclusion, that “the wax is solely known through the mind”.%%wcignore

Length: 600 words max.%%wcignore
\end{problem}

\begin{answer}
During René Descartes’ second meditation, the philosopher argued about the definition of a wax. In context, he was trying to define the meaning of things (bodies). His definition of a body until now is \emph{something which can be in a place, has a volume by itself, can be percieved through the senses, and is movable}.%54

He had started by trying to define who Himself was. Since it is too difficult to define a person, due to their complexity, he moved forward and tried to define a piece of wax.%34

He had chosen a piece of wax since he considered that he distinctively knew it, being a \emph{particular}~\nocite{unipaper} body which he \emph{could touch and see} ~\autocite{unipaper}. By giving the piece of wax different physical properties such as colour, figure and size; he learned after a while that these properties would not be permanent, but instead change over time ---due to the heat of the fire---.%62

Since the definition of the wax couldn't be given by the properties he had witnessed, Descartes wondered if we were still talking about the same piece of wax. He concluded that the piece we were talking about hadn't changed, but instead stated that one cannot define an item by the means of the senses. Sensorial perceptions would only explain how they appear to any given person at any given time.%49

After this conclusion, Descartes reapproached his methodological doubt, and asked himself: \emph{“What is left of a body after removingi ts physical traits”?}~\autocite{unipaper} Nothing. Only “something extended, flexible and mutable/changeable\footnote{I have used the words mutable/changeable as they make more sense, as \textcite{wrongword} explains.}”. By this, in order, he means that the figure occupies volume, can change its form, and can change location. He, nevertheless, wanted to formally define these terms beyond those descriptions.

Reflecting on the last two, flexible and mutable, he discovers the limitations of his mind. His imagination cannot grasp the idea of infinity, which is required to be able to see all the shapes the piece of wax can take. Due to this, he proves that the wax is not the product of his imagination, but something beyond. He makes no explicit statement here about what the imagination is, but it is well known \autocite{limitedimagination} that he believes imagination is a cognitively limited power, as it is part of the material world.

Since he cannot understand all the properties the piece of wax can have, his mind cannot either understand what the wax is in its entirety. This is even more evident if we try to talk about pieces of wax in general, instead of only considering a singular item.

\pagebreak
Now, \emph{if the mind cannot wrap itself around the true meaning of the piece of wax, what then does the mind think of when thinking about a piece of wax}---he wonders~\autocite{unipaper}
. The answer, and emphasis on this, it isn’t the senses which bring our knowledge of perception, but our intuition instead. He also claims we can improve this intuition of any given body through attentive thinking, and, again, methodological doubt.

To sum it all up, Descartes uses a piece of wax to explain the limitations of our mind, the properties of a body, the differences between permanent and temporary qualities of a defined body. He brings up our imagination to explain these, as our senses aren’t capable of learning anything “real”. That is, true for all cases; being universal truths.
\end{answer}

\vspace*{10ex}
\printbibliography

\end{document}
\endinput %%wcignore