\documentclass{../../myassignment}

\courselabel{IN1080}
\exercisesheet{Assignment 1}{Measuring Voltages}
\student{Rolf Vidar Hoksaas with Jonatan H. Hansen}

\newcommand{\ohm}{$\Omega$ }
\newcommand{\volt}{$V$ }
\newcommand{\amperes}{$A$ }
\newcommand{\percent}{$\%$}
\newcommand{\micro}{$\mu$}
\newcommand{\kilo}{$k$}

\begin{document}

	\begin{problem}
		Calculate the required parasitic resistance $R_W$ in one inductor (corresponding to the wire resistance between $A_1$ and $A_1\,’$ in Fig. 3, so that 2 inductors in series draws 1.2A from a single 5V voltage source.
		
	\end{problem}

	\begin{answer}
		$R_w = \frac{V_w}{I}$
	\end{answer}

	\begin{problem}
		Calculate the required length of copper wire in one inductor to get the resistance found in question 1 for a 0.4mm diam (AWG26) copper wire when the resistivity of copper is $\rho = 1.7\cdot10^{-8}$ \ohm m.
	\end{problem}

	\begin{answer}
	
	\end{answer}

	\begin{problem}
		Use the drill and fill one reel with copper wire, as shown in Fig. 4.  Grind the lacquer insulation off the ends (Fig.5 left).  Measure the resistance $R_W$ of your new inductor with RS-12.  Assume that the length of the wire in the coil is 15m.  Does this resistance correspond well with the value calculated inquestion 2?  You can not expect a perfect match, but make sure that the value is greater than about 2 \ohm; otherwise the current and the heat will be too high.
	\end{problem}

	\begin{answer}
	
	\end{answer}

\end{document}