\documentclass{myassignment} 
\usepackage[top=1cm]{geometry}
\courselabel{IN1000}
\exercisesheet{Assignment 7}{Spillelister og sanger}

\usepackage{graphicx}

\begin{document}
	\begin{problem}
		Lag en tegning, gjerne på papir, av datastrukturen i slik den ser ut etter at en spilleliste er lest inn fra filen musikk.txt. Du trenger ikke tegne mer enn to Sang-objekter. Bruk notasjonen fra forelesningene, med variabler som bokser, objekter som sirkler eller bokser med runde hjørner,og referanser som pilerfra en variabel til et objekt. Du kan tegne strenger som innhold i enkle variabler (selv om de egentlig er objekter), mens en liste tegnes som et objekt som refereres tilfra en listevariabel.
	\end{problem}\\
	\begin{answer}
		\includegraphics[width=\textwidth]{classstructure.pdf}
	\end{answer}

	\begin{problem}
		I klassen Spilleliste er det en instansvariabel som lagrer alle sangene i en liste. Nevn minst en, helst to årsaker til at en ordbok ikke egner seg like godt i dette tilfellet.
	\end{problem}\\
	\begin{answer}
		Alle objekter har sine egne implementasjoner, og inneholder informasjon som kan endres over tid. Hvis vi lagrer en kopi av denne informasjonen i en ordbok vil vi miste muligheten til å oppdage disse endringene etterhvert, og vi vil heller ikke kunne endre på den originale informasjonen om vi måtte gjøre det. 

		To forskjellige sanger kan ha samme artist eller samme tittel, så det finnes ingen logisk måte å generere denne ordboken uten å måtte generere unike IDer i tillegg. Hvis vi først gjør det, blir det mer komplisert å søke etter informasjon i ordboken. I værste fall kunne vi ha definert objektene som subclasses av en ordbok, hvis vi ønsket å bruke unike IDer som keys... men det er ikke nødvendig for denne oppgaven.
	\end{answer}

	\begin{problem}
		Klassen Spilleliste kunne hatt filnavn som parameter til konstruktøren, og lest inn spillelisten fra fil ved opprettelsen av et nytt Spilleliste-objekt. Ser du nøn fordel ved ikke å gjøre dette i konstruktøren?
	\end{problem}\\
	\begin{answer}
		Å ha en påtvunget parameter som filnavn ville skape problemer om vi ønsket å skape en tom Spilleliste for å så fylle den senere. Jeg ser ingen grunn til at vi ikke skulle ha mulighet til å gjøre dette, med å sette en default verdi som \texttt{None}, og sjekke om \texttt{\_\_init\_\_} faktisk er blitt kjørt med en ekte value. Om tilfelle, kan vi kjøre \texttt{self.lesFraFil} med samme verdi som konstruktøren.
	\end{answer}
\end{document}