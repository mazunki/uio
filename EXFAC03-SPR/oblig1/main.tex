\documentclass{myassignment}
\courselabel{EXFAC03-SPR}
\exercisesheet{Oblig1}{}
%\usepackage{tipa}
\usepackage{fontspec}
\setmainfont{Doulos SIL}
\usepackage{dirtree}

\begin{document}

	\section{Pragmatikk}

	\begin{problem}
	Man kan tenke seg mange forskjellige tolkninger av denne setningen, ut fra situasjonen den ytres i: «Hvis han kommer på festen, drar jeg med en gang». Foreslå to ulike tolkninger og gi grunner for tolkningsforslagene dine der du utdyper hva vi mener med pragmatisk mening sammenlignet med semantisk betydning. Bruk eksemplet til å forklare fagbegrepene deiksis og illokusjonær handling også.
	\end{problem}

	Hvor mer en studerer denne sentingen, hvor flere tolkninger kan man finne på. Fra setningen vet man ikke hvor taleren befinner seg når hun sier setningen, men man kan fort anta at hun ikke er på festen, siden hun ellers ville brukt stedsindikator-ordet «her», og ikke «på festen». De relative begrepene «her» og «der» er stedsdeiksis, på lik linje med at «han» er et personsdeiksis. Vi vet ikke hvem «han» er, annet enn at «han» er relativ til henne.

	Vi vet heller ikke hvilket forhold «han» og taleren har med hverandre. Hvis de har et godt forhold til hverandre kan man tolke setningen som at hun må skyndte seg til festen. Om de er fiender kan det bety at hun bare kommer til å være innom et lite øyeblikk, og så dra igjen.

	Det er ikke noen tvil om at setningen er lokusjonær: det hun sier gir mening, og er riktig norsk. Det kan være vanskelig å tolke den illokusjonære handlingen uten den kontekstuelle sammenhengen vi mangler, men det faktum at hun sier hun skal dra på festen avhengig av om «han» er der eller ei er kvalifeserer det som en illokusjonær handling.

	\section{Språklydlære}
	\begin{problem}
	Forklar grunnlaget for termene til å beskrive artikulasjonssteder for konsonanter, sånne begrep som apikal-alveolar, dorsal-velar osv.  Gi deretter en fonetisk transkripsjon av frasen «kortdistanseløperne og resten av gjengen» slik du mener du uttaler dette (oppgi dialektområde/språklig bakgrunn).
	\end{problem}

	Når vi uttaler konsonanter har vi en aktiv og en passiv komponent til å produsere lydene. Den aktive delen er enten underleppa, eller så er det tungen i alle sine posisjoner (fra spissen til ryggen). Den passive delen, som vi ikke kontrollerer, men bare holder støtte, varierer fra helt ute på leppene til helst nedderst i uvulaen. Disse posisjonene kommer som par, og blir kalt for aktiv-passive lyder. 

	Ofte snakker vi også om luftpasssasje, nasalitet, stemthet, mm.
	
	«kortdistanseløperne og resten av gjengen»

	/ , k o ɽ t , i s t a n s ɛ 2 l ø pʷ ɛ ɽ a n, o ɽ e s t e n:, a v æ ɽʲ n /

	PS: Jeg føler meg nokså usikker på dette her. Jeg er opprinnelig fra Mandal, men vokste opp i Spania. Har brukt komma for tonem 1, og tallet 2 for tonem 2. Fikk ikke \LaTeX til å oppføre seg.

	\pagebreak
	\section{Morfologi}
	\begin{problem}
	Vis den morfologiske oppbygningen til ordet «barndomserindringene» med et morfologisk tre, og forklar kort de begrepene du bruker. Si deretter hvilke grammatiske trekk ordet «varmt» har i setningen «Vannet er varmt». Lag til slutt egne eksempler hvor du bruker de grammatiske trekkene du kom fram til, ett ord for hvert trekk. Orda kan være fra ulike ordklasser.
	\end{problem}

	\dirtree{%
		.1 barndomserindringene (N), determinativ sammensettning.
		.2 barndoms (A) forledd adjektiv,
		.3 barndom (N).
		.4 barn (N), nominal.
		.4 dom (AF), bøyning N til N.
		.3 s (BF), kontaktassimilasjon.
		.2 erindringene (N) etterledd substantiv.
		.3 erindring (N).
		.4 erindre (V), nominal?.
		.4 -e (BF) apokope.
		.4 ing (AF), bøyning V til N.
		.3 en (BF), bestemt form.
		.3 e (BF), flertall.
	}
	I parantes ser vi at noen segmenter er nominale (N), verb (V), adjektiver (A), affikser (AF) eller bøyningsformativene (C). Bøyningsmorfemene viser til ordet sine egenskaper, affikser kan andre på klassen, mens assimilasjoner og apokoper betyr at vi legger til eller mister noen bokstaver i løpet av prosessen.

	Ordet «varmt» er et bestemt singulert adjektiv, brukt i en adjektivfrase.


	\section{Syntaks}
	\begin{problem}
	På norsk kan man si både «Bilen stoppa» og «Bilen ble stoppa». Forklar likheten og forskjellen mellom disse to måtene å uttrykke seg på ved hjelp av de fire begrepene transitivitet, argument, aktiv og passiv.	\end{problem}

	Forskjellen mellom setningene «Bilen stoppa» og «Bilen ble stoppa» er at den første setningen er sagt i aktiv form, mens den andre blir brukt i passiv form. Vi sier ingenting om årsaken til hvorfor bilen stoppet i den aktive formen, men i den passive formen kommer det implisitt frem at det var en bevisst handling som ble gjort, og ikke bare et resultat av en tilfeldighet eller ulykke.

	Når man bruker den passive formen "Bilen ble..." forventer vi at det skal komme noe mer. Verbet krever et argument, i dette tilfellet et verb i partisipp-form som sier noe om hvordan subjektet ble.

\end{document}
