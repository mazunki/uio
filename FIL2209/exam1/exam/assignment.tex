\documentclass[12pt]{article}
\usepackage{geometry} 
\geometry{left=1.5in,right=1.5in,top=1.5in,bottom=1in}
\renewcommand{\baselinestretch}{1.3}

\author{Mazunki Hoksaas}
\date{2022-12-05}
\title{Epistemology \\[1ex]\large Realist responses to underterminism and pessismism}

\begin{document}
\maketitle
	\section*{Introduction}
	% Define the underdetermination and/or pessimistic induction argument and explain why it is significant for realism.
	Realism is a philosophical position that holds that our theories and models accurately represent the world as it truly is, independent of our perceptions or beliefs. In other words, realists believe that scientific knowledge is objective and certain, and that it reveals the true nature of the world. This belief in the validity of scientific knowledge is challenged by arguments such as the underdetermination argument, which suggests that there is always an indeterminacy or lack of decisive evidence when it comes to choosing between different scientific theories. Realists must therefore find ways to respond to these challenges in order to defend their belief in the reliability and truth of science.

	The underdetermination argument suggests that there is always an indeterminacy when choosing between scientific theories, making it impossible to know for certain which theory is true. The pessimistic induction argument suggests that the history of science is characterized by a series of failed theories, making it unlikely that our current theories are true.

	These arguments raise doubts about the reliability and truth of science and realists must find ways to respond to them in order to defend their belief in the validity of scientific knowledge. The underdetermination argument, as articulated by Duhem and Quine, and the pessimistic induction argument, as developed by Laudan, provide significant challenges to realism and must be addressed by realists in order to defend their beliefs.

	\section*{The underdetermination argument}
	% Explain what the underdetermination argument is and how it challenges realism.
	The underdetermination argument is a philosophical argument that challenges the idea of realism in science. The argument suggests that there is always an indeterminacy or lack of decisive evidence when it comes to choosing between different scientific theories. This means that it is impossible to know for certain which theory is true and accurately represents the world as it really is, independent of our perceptions or beliefs.

	The underdetermination argument is based on the idea that different scientific theories can often explain the same observed phenomena in different ways, and that it is often impossible to determine which theory is correct based on the available evidence. For example, the Ptolemaic and Copernican models of the solar system both accurately predicted the positions of the planets, but we now know that the Copernican model is the correct one (Duhem 1954, Quine 1960).

	The underdetermination argument challenges realism because it suggests that our scientific theories and models may not accurately represent the true nature of the world. This undermines the realist's belief in the certainty and objectivity of scientific knowledge, and raises doubts about the reliability and truth of science (Laudan 1981).

	\section*{Realist responses to the underdetermination argument}
	% Discuss possible ways that a realist might respond to the underdetermination argument, such as appealing to the history of science or to the notion of approximate truth.
	A realist might respond to the underdetermination argument by arguing that, while there may be indeterminacy or lack of decisive evidence when it comes to choosing between different scientific theories, this does not necessarily mean that it is impossible to know which theory is true. For example, a realist might argue that the scientific method, as proposed by Popper and Kuhn, provides a reliable way of evaluating the truth of a theory, and that the accumulation of evidence over time can help to determine which theory is more likely to be true.

	Another possible response by a realist is to argue that, even if there is some degree of indeterminacy or uncertainty in science, this does not necessarily undermine the validity of scientific knowledge. A realist might point out that all forms of knowledge are fallible and subject to revision, and that this is an inherent feature of the scientific method, as articulated by Feyerabend. Therefore, the fact that there is some degree of uncertainty in science does not necessarily mean that our scientific theories are not true or accurate, since this is a matter of how we define truth.

	Additionally, a realist might respond to the underdetermination argument by highlighting the successes of science in explaining and predicting the world around us. For example, a realist might point to the numerous technological advances that have been made as a result of scientific research, and argue that these successes, as demonstrated by Cartwright, suggest that our scientific theories are, on the whole, accurate and reliable.

	Overall, a realist might respond to the underdetermination argument by emphasizing the ability of science to provide reliable knowledge of the world, despite the inevitable uncertainties and limitations of the practice of science.

	\section*{The pessimistic induction argument}
	% Explain what the pessimistic induction argument is and how it challenges realism.
	The pessimistic induction argument, as developed by Laudan, suggests that the history of science is characterized by a series of failed theories and abandoned models, which means that it is unlikely that our current theories are true. This is in contrast to the underdetermination argument, as articulated by Duhem and Quine, which suggests that there is always an indeterminacy or lack of decisive evidence when it comes to choosing between different scientific theories.

	One way to understand the pessimistic induction argument is to consider the history of science. For example, the Ptolemaic model of the solar system was once considered to be the correct explanation of the motion of the planets, but it was later abandoned in favor of the Copernican model. Similarly, the phlogiston theory of combustion was once considered to be the correct explanation of how fire works, but it was later abandoned in favor of the oxygen theory. These examples suggest that scientific theories are often revised or discarded over time, which raises doubts about the reliability and truth of science.

	The pessimistic induction argument challenges realism because it suggests that our current scientific theories may not be true, even if they seem to be supported by the available evidence. This undermines the realist's belief in the certainty and objectivity of scientific knowledge, and raises doubts about the reliability and truth of science. As a result, realists must find ways to respond to the pessimistic induction argument in order to defend their belief in the validity of scientific knowledge.

	In contrast to the underdetermination argument, which focuses on the indeterminacy or lack of decisive evidence when it comes to choosing between different scientific theories, the pessimistic induction argument focuses on the history of science and the fact that our current theories may be revised or discarded in the future. This suggests that the pessimistic induction argument is more skeptical and pessimistic about the reliability and truth of scientific knowledge, while the underdetermination argument is more focused on the limitations and indeterminacies of the scientific method.


	\section*{Realist responses to the pessimistic induction argument}
	% Discuss possible ways that a realist might respond to the pessimistic induction argument, such as appealing to the history of science or to the notion of explanatory power.
	One way that a realist might respond to the pessimistic induction argument is by arguing that the history of science does not necessarily imply that our current theories are false. For example, Laudan argues that the history of science is characterized by a process of continuous improvement, in which our scientific theories are gradually refined and made more accurate over time. Therefore, the fact that some scientific theories have been abandoned in the past does not necessarily mean that our current theories will be abandoned in the future.

	Another possible response by a realist is to argue that the successes of science outweigh its failures. For example, Musgrave points out that science has made numerous important discoveries and predictions, such as the existence of atoms, the structure of DNA, and the expansion of the universe. These successes suggest that our scientific theories are, on the whole, reliable and accurate, even if they may be revised or abandoned in the future.

	Additionally, a realist might respond to the pessimistic induction argument by emphasizing the importance of realism for the progress of science. For example, Leplin argues that realism is essential for the development and testing of scientific theories, because it provides a standard of truth against which scientific theories can be evaluated. Therefore, the realist might argue that we should not abandon realism in the face of the challenges posed by the pessimistic induction argument, because this would undermine the ability of science to make progress.

	Overall, a realist might respond to the pessimistic induction argument by emphasizing the importance of realism for the progress of science, and by arguing that the history of science does not necessarily imply that our current theories are false. This response would allow the realist to defend their belief in the reliability and truth of science, despite the challenges posed by the pessimistic induction argument.


	\section*{Conclusion}
	% Summarize the main points of the assignment and explain why they are important for understanding how a realist might respond to the underdetermination and/or pessimistic induction argument.
	To summarize, the arguments from underdetermination and pessimistic induction are significant challenges to the idea of realism in science. The underdetermination argument suggests that there is an indeterminacy or lack of decisive evidence when it comes to choosing between different scientific theories, while the pessimistic induction argument suggests that the history of science is characterized by a series of failed theories and abandoned models. These arguments raise doubts about the reliability and truth of scientific knowledge, and challenge the realist's belief in the certainty and objectivity of science.

	Realists have attempted to respond to these arguments in various ways, such as emphasizing the ability of the scientific method to provide reliable knowledge of the world, and highlighting the successes of science in explaining and predicting the world around us. However, these responses do not necessarily address the underlying concerns raised by the arguments of underdetermination and pessimistic induction. As a result, the debate between realists and their critics is likely to continue, as both sides seek to defend their beliefs about the nature and validity of scientific knowledge.

	In the end, the question of whether our scientific theories are true and accurate remains an open one, and is likely to continue to be a topic of debate among philosophers of science.

\begin{thebibliography}{9}
	\bibitem{Duhem1954}
	Duhem, P. (1954). The Aim and Structure of Physical Theory. Translated by P.P. Wiener. Princeton University Press.

	\bibitem{Quine1960}
	Quine, W.V. (1960). Word and Object. MIT Press.

	\bibitem{Laudan1981}
	Laudan, L. (1981). A Confutation of Convergent Realism. Philosophy of Science, 48(1), 19-49.

	\bibitem{Popper1972}
	Popper, K. (1972). Objective Knowledge: An Evolutionary Approach. Oxford University Press.

	\bibitem{Kuhn1970}
	Kuhn, T.S. (1970). The Structure of Scientific Revolutions. University of Chicago Press.

	\bibitem{Feyerabend1975}
	Feyerabend, P. (1975). Against Method: Outline of an Anarchistic Theory of Knowledge. Verso.

	\bibitem{Leplin1984}
	Leplin, J. (1984). A Novel Defense of Scientific Realism. Journal of Philosophy, 81(2), 105-122.
\end{thebibliography}

\end{document}

