\documentclass{myassignment}
\courselabel{FIL1002}
\exercisesheet{Oblig1}{Part 2}

	%\begin{problem}
	%	Trinn 2 er en individuell oppgave. Nå kan dere bruke Gettiers tekst i arbeidet, men husk at det er viktig å formulere poengene i eget språk, så langt det er mulig. Hver student skriver et essay, maks 500 ord, som utdyper innholdet i de fire setningene fra Trinn 1.
	%\end{problem}
\begin{document}

	I gruppearbeidet innledet vi hva tradisjonell kunnskap er, og nevnte såvidt hva Justified True Belief er. Denne tankegangen stammer opprinnelig fra Plato, og er lett beskrevet som en relasjon mellom en virkelighet, et subjekt som tror på denne virkeligheten og har tilstrekkelig grunnlag for denne troen.

	I dag er det vel kanskje litt mer aktuelt å definere kunnskap fra et litt mer vitenskapelig og metodisk perspektiv, der man har en teori om en virkelighet, som stammer fra rasjonaliteten, og som er testet empirisk for å være riktig på alle fortenkelige måter.

	Uansett kritiserer Gettier denne form for kunnskap, siden man aldri vet om det man tror man vet faktisk er slik man tror det er. Man kan kanskje tro at man har et godt holdepunkt ved noen premisser, men det er umulig å vite om man faktisk har rett. Vi kjenner jo ikke til det vi ikke har hørt om før. Med andre ord kan man bare tro ting, man kan ikke vite noe som helst. Goldman kritiserer dette, og mener at man faktisk kan ha kunnskap, dersom det viser seg at vi trodde rett, på rett grunnlag.

	Etterhvert kritiserer Goldman seg selv ved å rette på kausalitetsforholdet til kunnskapen, og sier istedenfor at man må forstå hele konteksten rundt den tingen man vet, altså alle detaljene som kan vrenge tilliten til vår tro.

	I forrige del avsluttet vi ved å spørre om forholdet hver person har til en ting vil avgjøre konteksten til de forskjellige kunnskapene om denne. Det er vanskelig å svare på dette spørsmålet, siden noen vil mene at en spade er en spade, mens andre kan tenke at gjenstanded ikke er en spade, men helst et kunstverk, og ville kanskje blitt fornærmet om jeg tok den i bruk til å grave jord med. Det er fort å tenke at dette kun er en lingvistisk forskjell, men det er jo åpenbart mer en bare en semantisk forskjell når kunsteren blir trist og frustrert om vi bruker gjenstanden som et verktøy.

	Betyr dette nødvendigvis at alt kan relativiseres? Vel, noen vil jo fort si \textit{«men 1+1 er jo 2, uansett hva du tror!!»}, men forståelsen vi har om tall og pluss-symbolet er jo egenlig kun en konvensjon som vi har tatovert inn i tankene våre. Så, i den forstand vi har av matte, så er det kanskje nyttig og pragmatisk i forhold til virkeligheten... men hva er forholdet mellom denne metodiske innbakningen og faktisk kunnskap? Det kan godt stemme at både tall og addisjon er begger er universaler, men det virker som om vi aldri vil kunne oppnå faktisk og fullstendig kunnskap om dem.

\end{document}
