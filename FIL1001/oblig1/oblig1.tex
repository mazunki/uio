\documentclass{article}
\usepackage{times}
\usepackage[margin=1.25in]{geometry}
\renewcommand{\baselinestretch}{1.5}

\usepackage[
    style=authoryear-icomp,
    natbib
]{biblatex}
\addbibresource{biblio.bib}

\usepackage[UKenglish]{babel}
\usepackage[autostyle]{csquotes}
\MakeOuterQuote{"}
\newcommand{\q}[1]{``#1''}

\usepackage{enumitem}
\setlist{parsep=0pt,listparindent=\parindent,}

\begin{document}
	\begin{flushright}
		\textit{University of Oslo}
		\\\textit{Mazunki Hoksaas}
		\\\textbf{FIL1001}, \textit{2021-09-27}
	\end{flushright}

	\section*{Oblig1}
	
	\textit{Er det noen gode beviser for Guds eksistens? Diskuter med utgangspunkt i Anselms ontologiske bevis og Thomas Aquinas´ 5 veier.}

	\subsection*{Anselm}
	Anselm was an Italian philosopher and monk from the Holy Roman Empire. He was buried in Canterbury, where he held the title of Archbishop of Canterbury. He eventually became a saint. Sufficient to say, he was quite a remarkable theologician.

	In the \citetitle{anselm1078}, he provides an ontological argument to explain the existence of God. He wrote this book for fellow believers, to provide a rational basis in which one could understand the existence of God, but has also been used to convince (or try to, at least) atheist rationalists to become convinced. His books mainly contains two arguments, both of which follow logical conclusions, paraphrased below:
	
	\subsubsection*{Argument 1, in chapter 2}
	\begin{itemize}
		\item We know the definition for God is: ``There is no better being than God''.
		\item The idea of this definition exists in our head.
		\item It is better to be real than to not be real.
		\item Following, we can imagine something better than a non-real God.
		\item Thus a real God must exist.
	\end{itemize}

	Aquinas (\citeyear{aquinas1485}), which we will talk more about in the next section, rejects this argument. He explains that we are mere mortals, unable to conceive the concept of God in its entirety. Only Him is big enough to be able to graps the reality of something as big as Himself.

	\subsubsection*{Argument 2, in chapter 3}
	\begin{itemize}
		\item We have a definition for God: ``There is no better being than God''.
		\item A being which must necessarily exist is better than one who may not exist.
		\item If you can think of God in a way in which he's not necessary... you're doing it wrong.
		\item God must exist.
	\end{itemize}

	Immanuel Kant (\citeyear{kant1781}) has an argument which slamdunks Anselm's necessity-argument real hard. To use Kant's example, a triangle is known to have three angles, by definition. While this is necessarily true for all instances of triangles, to say that this a universal necessity implies that we know already that triangles do exist.

	In other words, just because ``God would need to be necessary, if they exist'' doesn't necessarily imply that this necessity exists, as that would depend upon the existence of God. Isn't that what we're trying to prove?
	

	\subsection*{Thomas Aquinas}
	In \cite{aquinas1485}, Thomas Aquinas explains that everyone is capable of reasoning about God, because it's a natural thing from the world, and we have countless examples of pagan authors talking about God, as a concept, without ever having read the bible. While some of them might make mistakes, sufficient studying and cognitive thinking will allow them to offer philosphical answers to whatever question they get asked, he claims.

	Aquinas also explains how we can have an understanding of something without ever having seen the thing itself with a proof-by-example: By seeing smoke, we understand that there must be a fire nearby, without ever having seen the source of this fire. According to him, this is a fundamental way of how we use our reasoning, and claims there is a seed which grows over time, allowing us to make more inferences like the one mentioned. This seed can develop positively, but it can also be corrupted in several ways.

	\subsubsection*{Prima Via: The Unmoved Mover}
	Aquinas understands change as a transit from potentiality to actuality, meaning things don't really change their essence or substance, we would still be talking about the same thing. Changing from non-existing to existing would be a change in potentiality, as it doesn't make sense to say a non-existing thing has the potential to be a thing.

	Since we can see things changing, and all these are being changed by something else, we can essentially confirm there are things which move other things. Since Aquinas doesn't believe in the infinite regression (which is a Fair Assumption), this leads to the regression of movers being finite. There must be a starting point. This starting point can be no other being than God, the Unmoved Mover, the Unchanged Changer.
	

	\subsubsection*{Secundia Via: The First Cause}
	Similarly to the previous argument, we have a regression of causes. Everything which is caused, must have been caused by something, and this something can never be itself. That would be a contradiction in and of itself, because who would have caused this thing to act this way? Aquinas is not talking about a sequential events, but rather about the efficient agent which \citeauthor{aristotlecauses} has talked about.

	There essentially must be a first cause for everything else, and thus God exists.


	\subsubsection*{Tertia Via: Contingency}
	We observe things which could be possible, but are not necessary. This is also true of Nothingness. It could theoretically be the case that Nothing existed. Furthermore, he claims, everything comes and goes over time, so why wouldn't there be a time when Nothing existed?
	
	Seeing as we exist now, and nothing can spawn from Nothing, there must surely be some being controlling it, or some being which is superior to the realm of where Nothingness can be. This must be God.


	\subsubsection*{Quarta Via: Degree}
	Remember Platon's idea of virtue? One could be better or worse at any said skill? What if you were just Perfect at every single attribute? You would even be the best Exister in existence. This, by definition, is God. He's even better at being Real than you!


	\subsubsection*{Quinta Via: Final Cause}
	We can observe intelligent beings act with intention, we seem to act with a purpose. This cannot be entirely random, as this pattern is consistent even among a multitude of beings without reason. Since a goal for our existence (survival?),  exists, this must be set by a cause. This cause is none other than God.

	\printbibliography{}

\end{document}
